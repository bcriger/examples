\documentclass[12pt,a4paper]{article}
\usepackage{fontspec}
\defaultfontfeatures{Mapping=tex-text}
\usepackage{xunicode}
\usepackage{xltxtra}
\usepackage{amsmath}
\usepackage{amsfonts}
\usepackage{amssymb}
\usepackage{amsthm}
\usepackage{bbold}
\usepackage{enumitem}
\usepackage{graphicx}
\usepackage[left=1.5cm,right=1.5cm,top=1.5cm,bottom=1.5cm]{geometry}
\usepackage{nicefrac}
\usepackage{xspace}

\providecommand{\id}{\mathbb{1}}
\providecommand{\gcdd}{\mathrm{gcd}}
\providecommand{\lcm}{\mathrm{lcm}}
\providecommand{\FAG}{finite abelian group\xspace}
\providecommand{\bemph}[1]{\emph{\textbf{#1}}}
\providecommand{\abs}[1]{\left \vert #1 \right \vert}
\providecommand{\index}[2]{\left[ #1 \,:\, #2 \right] #1 \right \vert}
\providecommand{\hint}{\emph{Hint: }}
\providecommand{\soln}{\textbf{Solution: }}
\providecommand{\set}[1]{\left \lbrace #1 \right \rbrace}
\providecommand{\st}{such that\xspace}
\providecommand{\JK}{Joel Klassen\xspace}
\providecommand{\CV}{Christophe Vuillot\xspace}
\providecommand{\JH}{Jonas Helsen\xspace}
\providecommand{\LB}{Lennart Bittel\xspace}
\renewcommand{\iff}{\textrm{if and only if}\xspace}
\providecommand{\inv}{^{-1}}
\providecommand{\gen}[1]{\left \langle #1 \right \rangle}
\providecommand{\hism}{homomorphism\xspace}
\providecommand{\iism}{isomorphism\xspace}
\providecommand{\mod}{\textrm{ mod }}

\title{Physics of Electric Propulsion}
\author{Robert G. Jahn}

\begin{document}
\maketitle
Second Printing
\section*{Problems}
\subsection*{Chapter 1}
\paragraph*{1-1}
Derive Equation 1-1.

\soln Equation 1-1 is describes the balance of forces on a rocket exerting thrust:
\begin{equation}
	m\dot{v} = \dot{m}u_e + F_g
\end{equation}
We can get this one from Newton's second law for an object which has variable mass:
\begin{flalign}
	F &= \dfrac{dp}{dt} \\ 
	F_g &= \dot{m}u_e + m\dot{v}
\end{flalign}
There's probably a minus sign in there, but I'm not too bothered.

\paragraph*{1-2}
Verify the first three characteristic velocities quoted in Table 1-1. 
Explain why the values for gentle spiral ascent exceed those for impulsive ascent for a given Earth orbit transfer.

\soln I'm guessing the Oberth effect is the reason for the sub-optimality of gentle spiral ascent.
It could also be $g \Delta t$ loss. 

\paragraph*{1-3}
For a mission characterised by a velocity increment $\Delta v$ and a rocket of constant thrust level $T$, derive an expression for the optimum specific impulse in terms of power supply specific mass $\alpha$ and the initial total mass of the spacecraft. 

\soln This is covered in Section 1-4, ``The Power Supply Penalty''. 
Let's start by re-deriving the $u_e$ which optimises the payload mass ratio for fixed $\Delta t$ and $T$.
%\begin{flalign}
%	\dfrac{m_u + m_p + \Delta m}{m_u}
%\end{flalign}
We assume a linear relationship between plant power, plant mass, and thrust:
\begin{equation}
	m_p = \alpha P = \dfrac{\alpha T u_e}{2 \eta}
\end{equation}
From kinematics, we also know
\begin{equation}
	\Delta m = \dfrac{T \Delta t}{u_e}
\end{equation}
Both of these expressions depend only on the variable of interest ($u_e$), and parameters which are independent of $u_e$, namely $T$, $\Delta t$ (fixed by assumption), $\alpha$ and $\eta$ (efficiencies intrinsic to their related devices). 

We'd like to minimise $m_p + \Delta m$:
\begin{flalign}
	\frac{d}{d u_e} (m_p + \Delta m) &= \frac{d}{d u_e} \left( \dfrac{\alpha T u_e}{2 \eta} + \dfrac{T \Delta t}{u_e} \right) \\ 
	& = \dfrac{\alpha T}{2 \eta} - \dfrac{T \Delta t}{u_e^2} = 0
\end{flalign}
This results in a minimum at $u_e = \sqrt{\dfrac{2 \eta \Delta t}{\alpha}}$, since the second derivative is positive at that point (being a sum of positive terms).

Now let's try the same thing, but with $T$ and $\Delta v$ fixed. 
Our expression for $m_p$ still only depends on fixed quantities and $u_e$, but our expression for $\Delta m$ depends on $\Delta t$, which in turn depends on $u_e$. 
We could also express $\Delta m$ using the Tsiolkovsky equation:
\begin{equation}
	\Delta m = m_0 \left( 1 - e^{-\frac{\Delta v}{u_e}} \right),
\end{equation}
but this still contains an $m_0$, which depends on $u_e$ for fixed $T$ and $\Delta v$. 
Let's substitute again:
\begin{flalign}
	m_0 &= (m_u + m_p)e^{\frac{\Delta v}{u_e}} \\
	\Delta m &= (m_u + m_p) \left( e^{\frac{\Delta v}{u_e}} - 1 \right)
\end{flalign}
This leads us to the same place as if we had just written down
\begin{flalign}
	m_0 &= \left( m_u + \dfrac{\alpha T u_e}{2 \eta} \right) e^{\frac{\Delta v}{u_e}} \\
	\frac{m_u}{m_0} &= r = \frac{m_u e^{-\frac{\Delta v}{u_e}}}{m_u + \dfrac{\alpha T u_e}{2 \eta}}\\
	\dfrac{\partial r}{\partial u_e} &= \frac{ m_u e^{-\frac{\Delta v}{u_e}} \left( m_u \Delta v + \frac{\alpha T u_e (\Delta v - u_e)}{2\eta} \right) }{u_e^2 \left( m_u + \frac{\alpha T u_e}{2 \eta} \right)^2} = 0 \\
	\frac{2\eta m_u \Delta v}{\alpha T} + \Delta v u_e - u_e^2  &= 0 
\end{flalign}
There's only one positive root of this quadratic equation:
\begin{equation}
	u_e = \frac{\Delta v}{2} \left( 1 + \sqrt{\frac{8 \eta m_u}{\alpha T \Delta v} + 1} \right)
\end{equation}
Not too bad-looking, but who knows if it's correct?
\paragraph*{1-4}
If the mass delivered by a rocket, $m_f$, is separated into useful payload $m_u$ and the power-producing system mass $m_p = \alpha P$, show that equation 1-6 takes the form
\begin{equation}
	\frac{m_u}{m_0} = e^{ - \frac{\Delta v}{u_e}} - \dfrac{\alpha u_e^2}{2 \eta \Delta t} \left( 1 - e^{ - \frac{\Delta v}{u_e}} \right)
\end{equation}
where $\Delta t$ is the thrusting time and $\eta$ is the conversion efficiency. 
What is the optimal $u_e$ for this mission?

\soln Let's start with the Tsiolkovsky equation (1-6 in the book)
\begin{flalign}
	\frac{m_f}{m_0} &= e^{-\frac{\Delta v}{u_e}} \\
	\frac{m_u + m_p}{m_0} &= e^{-\frac{\Delta v}{u_e}} \\
	\frac{m_u}{m_0} &= e^{-\frac{\Delta v}{u_e}} - \frac{m_p}{m_0} \\
	&= e^{-\frac{\Delta v}{u_e}} - \frac{\alpha P}{m_0} \\
	&= e^{-\frac{\Delta v}{u_e}} - \frac{\alpha T u_e}{2 \eta m_0} 
\end{flalign}
To round this one out, we need the total impulse relation:
\begin{equation}
	T \Delta t = u_e \Delta m
\end{equation}
We can start again from the Tsiolkovsky equation to put $\Delta m$ in the appropriate form:
\begin{flalign}
	\frac{m_0 - \Delta m}{m_0} &= e^{-\frac{\Delta v}{u_e}}\\
	\frac{m_0}{m_0} - \frac{m_0 - \Delta m}{m_0} &= 1 - e^{-\frac{\Delta v}{u_e}}\\
	\Delta m &= m_0 \left( 1 - e^{-\frac{\Delta v}{u_e}} \right)\\
	\therefore T &= \dfrac{m_0 u_e}{\Delta t} \left( 1 - e^{-\frac{\Delta v}{u_e}} \right)
\end{flalign}
Substitution solves the problem. 

Let's differentiate to find the maximum $\frac{m_u}{m_0}$:
\begin{flalign}
	\frac{d}{du_e} \left( \frac{m_u}{m_0} \right) &= \frac{e^{-\frac{\Delta v}{u_e}} \Delta v  }{u_e^2} - \frac{ \left( 1 - e^{-\frac{\Delta v}{u_e}} \right) \alpha u_e}{ \eta \Delta t } + \frac{ e^{-\frac{\Delta v}{u_e}} \alpha \Delta v}{ 2 \eta \Delta t } = 0
\end{flalign}
This equation looks pretty much unsolvable. 

\end{document}

[label=(\roman*)]